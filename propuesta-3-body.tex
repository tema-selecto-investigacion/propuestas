\hspace{-0.6cm}
\fbox{
  \parbox{\textwidth}{
      Discutir con el profesor el tema y evaluar posibles sistemas y/o arquitecturas candidatas para esto
  }
}


\section{Problema}
El no entender el impacto de decisiones en el diseño de un sistema puede ser costoso y riesgoso. Probar  software significa que ya se ha hecho un esfuerzo en su implementación. Por ejemplo, si las pruebas revelan problemas de rendimiento, es probable que la arquitectura necesite ser cambiada, lo que puede involucrar costos adicionales. Estos costos surgen porque en los sistemas empresariales de software de hoy en día, bajos rendimientos son principalmente el efecto de una arquitectura no apropiada en lugar de mal código.

\section{Justificación}
En otras ramas de la ingeniería, es una práctica común simular un modelo de un artefacto antes de ejecutarlo. Diseños de modelos de autos, circuitos electrónicos, puentes, entre otros, son simulados para entender el impacto de decisiones de diseño de varios atributos de calidad de interés tales como seguridad, consumo de energía o estabilidad. La habilidad de predecir las propiedades de un artefacto en base a su diseño sin necesariamente implementarlo, es una de las características centrales de una disciplina de ingeniería. Desde este punto de vista de las disciplinas de ingeniería establecidas, la ``ingeniería de software'' es apenas una disciplina de ingeniería. Muy a menudo, los ingenieros de software carecen del entendimiento del impacto de decisiones de diseño en atributos asociados con la calidad del sistema como rendimiento o confianza. Como resultado, ellos intentan probar la calidad en costosos ciclos de prueba-y-error.

\section{Objetivo General}
Proponer diseños alternativos de un \texttt{<Sistema y/o Arquitectura por definir>} a partir de su rendimiento por medio de modelaje y simulación.