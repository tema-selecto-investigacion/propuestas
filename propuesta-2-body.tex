\hspace{-0.6cm}
\fbox{
  \parbox{\textwidth}{
    \textbf{Idea tomada de: (Ver documento al final)}\\
    \url{https://sdqweb.ipd.kit.edu/wiki/Modeling_and_Simulation_of_Distributed_Message_Queues}
  }
}

\section{Problema}
No se tiene cuantificado el impacto en el rendimiento del uso de sistemas distribuidos de intercambio de mensajes en un sistema de software. 

\section{Justificación}
En general, la influencia del impacto de la agregación y/o modificación de componentes en un sistema de software es usualmente una tarea que no se realiza. Muchas veces se asume que el delegar funciones a un producto o librería de código va a brindar resultados positivos de manera casi inmediata. No se toma en cuenta que cada nuevo componente que se agrega al sistema también aporta complejidad y comporatmiento los cuales podrían comprometer un sistema.\\ 

Los sistemas distribuidos de intercambio de mensajes se han convertido en una alternativa para alivianar la carga y reducir la complejidad de un sistema de software. Por medio de estas, los sistemas se encargan de publicar mensajes a una cola mientras que por otro lado otros sistemas se encargan de leer los mensajes publicados a esa cola periódicamente.\\

Una investigación sobre la influencia del impacto en el rendimiento del uso de  sistemas distribuidos de mensajes en un sistema de software puede ayudar a dar a conocer factores favoren o desfavorecen el uso de los mismos y, además de esto, podría representar un marco de referencia inicial por medio del cual se pueda evaluar la adopción de estas tecnologías \emph{a priori}. 

\section{Objetivo General}
Evaluar la influencia en el rendimiento de sistemas distribuidos de intercambio de mensajes por medio de modelado y simulación

\label{sec:original}
\includepdf[pages=1]{modeling-simulating-distributed-queues}