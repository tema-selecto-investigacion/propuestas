\documentclass[format=acmsmall, review=false, screen=true]{acmart}


\usepackage{booktabs} % For formal tables
\usepackage[utf8]{inputenc}
\usepackage[T1]{fontenc}
\usepackage[spanish]{babel}
\usepackage{pdfpages}

\usepackage[ruled]{algorithm2e} % For algorithms
\renewcommand{\algorithmcfname}{ALGORITHM}
\SetAlFnt{\small}
\SetAlCapFnt{\small}
\SetAlCapNameFnt{\small}
\SetAlCapHSkip{0pt}
\IncMargin{-\parindent}


% Metadata Information
\acmJournal{TEMA}
%\acmVolume{9}
%\acmNumber{4}
\acmArticle{\textbf{Propuesta 3}}
\acmYear{2018}
\acmMonth{2}
\copyrightyear{2018}
%\acmArticleSeq{9}

% Copyright
\setcopyright{none}
%\setcopyright{acmlicensed}
%\setcopyright{rightsretained}
%\setcopyright{usgov}
%\setcopyright{usgovmixed}
%\setcopyright{cagov}
%\setcopyright{cagovmixed}

\settopmatter{printacmref=false}
\renewcommand\footnotetextcopyrightpermission[1]{} % removes footnote with conference information in first column
%\pagestyle{plain} % removes running headers


% DOI
%\acmDOI{0000001.0000001}

% Paper history
%\received{February 2007}
%\received[revised]{March 2009}
%\received[accepted]{June 2009}


% Document starts
\begin{document}
% Title portion. Note the short title for running heads
\title[A Multifrequency MAC for Wireless Sensor]{Modelado y Simulación de un \texttt{<Sistema y/o Arquitectura por definir>}
  \\ \small{Instituto Tecnológico de Costa Rica\\ Escuela de Ingeniería en Computación\\ Maestría en Computación\\ MC-7205 Tema Selecto de Investigación\\}
}

\author{Carlos Martín Flores González}
\affiliation{%
  \institution{Carné: 2015183528}
%  \city{Cartago}
%  \state{Cartago}
%  \postcode{506}
%  \country{2015183528}
}
\email{mfloresg@gmail.com}

\authorsaddresses{}

%\begin{abstract}
%Multifrequency media access control has been well understood in
%general wireless ad hoc networks, while in wireless sensor networks,
%researchers still focus on single frequency solutions. In wireless
%sensor networks, each device is typically equipped with a single
%radio transceiver and applications adopt much smaller packet sizes
%compared to those in general wireless ad hoc networks. Hence, the
%multifrequency MAC protocols proposed for general wireless ad hoc
%networks are not suitable for wireless sensor network applications,
%which we further demonstrate through our simulation experiments. In
%this article, we propose MMSN, which takes advantage of
%multifrequency availability while, at the same time, takes into
%consideration the restrictions of wireless sensor networks. Through
%extensive experiments, MMSN exhibits the prominent ability to utilize
%parallel transmissions among neighboring nodes.
%\end{abstract}


%
% The code below should be generated by the tool at
% http://dl.acm.org/ccs.cfm
% Please copy and paste the code instead of the example below.
%
%\begin{CCSXML}
%<ccs2012>
% <concept>
%  <concept_id>10010520.10010553.10010562</concept_id>
%  <concept_desc>Computer systems organization~Embedded systems</concept_desc>
%  <concept_significance>500</concept_significance>
% </concept>
% <concept>
%  <concept_id>10010520.10010575.10010755</concept_id>
%  <concept_desc>Computer systems organization~Redundancy</concept_desc>
%  <concept_significance>300</concept_significance>
% </concept>
% <concept>
%  <concept_id>10010520.10010553.10010554</concept_id>
%  <concept_desc>Computer systems organization~Robotics</concept_desc>
%  <concept_significance>100</concept_significance>
% </concept>
% <concept>
%  <concept_id>10003033.10003083.10003095</concept_id>
%  <concept_desc>Networks~Network reliability</concept_desc>
%  <concept_significance>100</concept_significance>
% </concept>
%</ccs2012>
%\end{CCSXML}
%
%\ccsdesc[500]{Computer systems organization~Embedded systems}
%\ccsdesc[300]{Computer systems organization~Redundancy}
%\ccsdesc{Computer systems organization~Robotics}
%\ccsdesc[100]{Networks~Network reliability}

%
% End generated code
%

%
%\keywords{Wireless sensor networks, media access control,
%multi-channel, radio interference, time synchronization}




\maketitle

% The default list of authors is too long for headers.
%\renewcommand{\shortauthors}{G. Zhou et al.}

\hspace{-0.6cm}
\fbox{
  \parbox{\textwidth}{
      Discutir con el profesor el tema y evaluar posibles sistemas y/o arquitecturas candidatas para esto
  }
}


\section{Problema}
El no entender el impacto de decisiones en el diseño de un sistema puede ser costoso y riesgoso. Probar  software significa que ya se ha hecho un esfuerzo en su implementación. Por ejemplo, si las pruebas revelan problemas de rendimiento, es probable que la arquitectura necesite ser cambiada, lo que puede involucrar costos adicionales. Estos costos surgen porque en los sistemas empresariales de software de hoy en día, bajos rendimientos son principalmente el efecto de una arquitectura no apropiada en lugar de mal código.

\section{Justificación}
En otras ramas de la ingeniería, es una práctica común simular un modelo de un artefacto antes de ejecutarlo. Diseños de modelos de autos, circuitos electrónicos, puentes, entre otros, son simulados para entender el impacto de decisiones de diseño de varios atributos de calidad de interés tales como seguridad, consumo de energía o estabilidad. La habilidad de predecir las propiedades de un artefacto en base a su diseño sin necesariamente implementarlo, es una de las características centrales de una disciplina de ingeniería. Desde este punto de vista de las disciplinas de ingeniería establecidas, la ``ingeniería de software'' es apenas una disciplina de ingeniería. Muy a menudo, los ingenieros de software carecen del entendimiento del impacto de decisiones de diseño en atributos asociados con la calidad del sistema como rendimiento o confianza. Como resultado, ellos intentan probar la calidad en costosos ciclos de prueba-y-error.

\section{Objetivo General}
Proponer diseños alternativos de un \texttt{<Sistema y/o Arquitectura por definir>} a partir de su rendimiento por medio de modelaje y simulación.


\end{document}
